%%%%%%%%%%%%%%%%%%%%%%%%%%%%%%%%%%%%%%%%%%%%%%%%%%%%%%%%%%%%%%%%%%%%%%%%%%%
% FILE    : Division.tex
% SUBJECT : An algorithm for multi-precision division.
% AUTHOR  : (C) Copyright 2014 by Peter C. Chapin <PChapin@vtc.vsc.edu>
%
%%%%%%%%%%%%%%%%%%%%%%%%%%%%%%%%%%%%%%%%%%%%%%%%%%%%%%%%%%%%%%%%%%%%%%%%%%%

% ++++++++++++++++++++++++++++++++
% Preamble and global declarations
% ++++++++++++++++++++++++++++++++
\documentclass{article}

% --------
% Packages
% --------
\usepackage{amssymb}
\usepackage{amsmath}
\usepackage{color}
\usepackage{fancyvrb}
\usepackage[pdftex]{graphicx}
\usepackage{listings}
\usepackage{hyperref}
\usepackage{url}

% The following are settings for the listings package
\lstset{language=Ada,
        basicstyle=\small,
        stringstyle=\ttfamily,
        commentstyle=\ttfamily,
        xleftmargin=0.25in,
        showstringspaces=false}


% ------------------
% Layout adjustments
% ------------------
% Avoid changing the layout; it is usually set via the document class (or in packages).
%
%\pagestyle{headings}
%\setlength{\parindent}{0em}
%\setlength{\parskip}{1.75ex plus0.5ex minus0.5ex}


% ------
% Macros
% ------

\newcommand{\code}[1]{\texttt{#1}}
\newcommand{\command}[1]{\texttt{#1}}
\newcommand{\filename}[1]{\texttt{#1}}
\newcommand{\newterm}[1]{\textit{#1}}

\long\def\pnote#1{\marginpar{\color{blue}{PC}}{\small \ \ $\langle\langle\langle$\
{#1 - PC}\
    $\rangle\rangle\rangle$\ \ }} 

% This macro is for text figures (program listings). Use as follows:
% \begin{figure}[htbp]
% \begin{textbox}{3in}  % The width of the text.
% \begin{Verbatim}
% ...
% \end{Verbatim}
% \end{textbox}
% \caption{...}
% \label{...}
% \end{figure}
%
\newsavebox{\savebigbox}
\newenvironment{textbox}[1]{
  \begin{lrbox}{\savebigbox}
  \begin{minipage}{#1}
  \vspace{1.0em}
}
{
  \vspace{0.50em}
  \end{minipage}
  \end{lrbox}\framebox[\textwidth]{\hfill\fbox{\usebox{\savebigbox}}\hfill}
}



% +++++++++++++++++++
% The document itself
% +++++++++++++++++++
\begin{document}

% ----------------------
% Title page information
% ----------------------
\title{Multiprecision Division}
\author{Peter C. Chapin\thanks{PChapin@vtc.vsc.edu}\\
  Vermont Technical College}
\date{September 7, 2014}
\maketitle

% ------------
% Main Content
% ------------

\section{Introduction}
\label{sec:introduction}

Cryptographic algorithms, particularly public key algorithms, often need to manipulate integers
that are much larger than the natural word size of the processor. These \textit{multiprecision}
operations are done with specialized subprograms, sometimes using special, high efficiency
algorithms.

This document describes a ``classic'' multiprecision division algorithm of the sort taught to
grade school children except cast in a precise mathematical form. The algorithm presented here
is essentially the same as presented in \cite{knuth-seminumerical}. I have provided additional
commentary relevant to ACO's use of this algorithm in its support for elliptic curve
cryptography.

\section{Inputs and Outputs}
\label{sec:inputs-outputs}

We consider only natural numbers (non-negative integers). ACO does not currently require the
division of signed numbers. If that becomes necessary the sign can be managed separately and so
need not affect the algorithm presented here.

We will represent our numbers in a base $b$ where $b$ is any convenient value, typically a power
of two. The size of $b$ should be chosen to facilitate computation by being a natural size for
the underlying machine. Here ``natural size'' means a size that fits into the machine's
registers and for which the machine supports primitive arithmetic operations. For example if
eight bit primitive operations are to be used, then $b = 2^8 = 256$.

A large value of $b$ such as $b = 2^{16}$ or $b = 2^{32}$ might result in faster computations by
allowing more bits of the number to be processed at once (thus reducing the number of times
various loops execute). However, in the examples that follow I will assume $b = 256$.

If $b$ is a power of two then a large binary number can be written as a sequence of base $b$
digits by just partitioning the bits of that number into groups where each group contains
$\log_2 b$ bits. For example if $b = 256$ then each ``digit'' of the number is a group of eight
bits. Thus a 192~bit number has 24 base 256 digits and a 384~bit number has 48 base 256 digits.

The algorithm below describes how to compute $u/v$ where $v$ can be written as a sequence of $n$
base $b$ digits $(v_{n-1}\ldots v_1 v_0)$ and where $u$ can be written as a sequence of
$m + n$ base $b$ digits $(u_{m+n-1}\ldots u_1 u_0)$. The algorithm requires that the most
significant digit of $v$, $v_{n-1}$, be non-zero.

In our case we are interested in dividing a 384~bit (48~digit) dividend by a 192~bit (24~digit)
divisor. However, the first step will be finding the most significant non-zero digit of the
divisor; it might not be digit 23. For example, suppose that the divisor is really a 20 digit
number (with leading zeros to fill it out to 192 bits). In that case $n = 20$ and we can write
$48 = m + 20$ giving $m = 28$. The quotient is at most an $m + 1$ digit number $(q_m
q_{m-1}\ldots q_0)$ and the remainder is at most an $n$ digit number $(r_{n-1}\ldots r_1 r_0)$.

The ACO library allows numbers of arbitrary sizes (up to 8192~bits) in 8~bit units. However, the
current implementation of division requires the dividend to be twice the length of the divisor
(not considering the possibility of leading zero digits). This is enforced by a precondition.

Notice that $n$ must be at least one since $n = 0$ implies that the divisor is zero which is
undefined.

\section{Algorithm}

The following algorithm given by Knuth \cite{knuth-seminumerical} as ``Algorithm D'' requires
that the number of significant digits in the divisor be strictly greater than one. The case
where the divisor is a single digit must be handled by a separate (simpler) algorithm.

Proceed as follows.

\begin{enumerate}

\item \textbf{D0.} Compute $n$ and $m$. Examine the divisor $v$ and find the most significant
  non-zero digit. This is digit $n - 1$ (the least significant digit is digit zero). Compute $m$
  from the formula $L = m + n$ where $L$ is the length of the dividend in digits. If $n = 0$
  return a divide by zero error indication.

\item \textbf{D1.} Normalize. Find the number of bit positions $v_{n-1}$ must be shifted to the
  left to move a one bit into the most significant position of $v_{n-1}$. Call this shift
  distance $s$. Set $d \leftarrow 2^s$ and multiply $u$ and $v$ by $d$.

  This multiplication can be accomplished by doing a left shift of $s$ bits in every digit of
  $u$ and $v$, starting from the least significant digit, taking care to move bits properly
  across the digit boundaries. The value in $v$ will remain $n$ digits long since $s$ was chosen
  to prevent it from overflowing, but the value in $u$ might require an additional digit.

\item \textbf{D2.} Initialize $j$. Set $j \leftarrow m$.

\item \textbf{D3.} Calculate $\hat{q}$, the estimated quotient digit. Set $\hat{q} \leftarrow
  \lfloor(u_{j+n}b + u_{j+n-1})/v_{n-1}\rfloor$ and set $\hat{r} \leftarrow (u_{j+n}b +
  u_{j+n-1}) \mod v_{n-1}$. Test: if $\hat{q} = b$ or $\hat{q}v_{n-2} > b\hat{r} + u_{j+n-2}$
  then $\hat{q} \leftarrow \hat{q} - 1$ and $\hat{r} \leftarrow \hat{r} + v_{n-1}$. Repeat the
  test (just once, not in a loop) if it results in $\hat{r} < b$.

\item \textbf{D4.} Multiply and subtract. Replace $(u_{j+n}u_{j+n-1}\ldots u_j)$ by
$$
(u_{j+n}u_{j+n-1}\ldots u_j) - \hat{q}(0v_{n-1}\ldots v_1 v_0)
$$
If the result is negative let the computation ``wrap around'' but remember that a borrow was
necessary.

\item \textbf{D5.} Test remainder. Set $q_j \leftarrow \hat{q}$. If a borrow was necessary in
  D4, go to D6, otherwise go to D7.

\item \textbf{D6.} Add back. $q_j \leftarrow q_j - 1$. Add $(0v_{n-1}\ldots v_1 v_0)$ into
  $(u_{j+n}u_{j+n-1}\ldots u_j)$. Ignore the carry that results (it undoes the earlier borrow).
  The probability that this step is needed is approximately $2/b$ so special test cases are
  needed to exercise it.

\item \textbf{D7.} Loop on $j$. Set $j \leftarrow j - 1$. If $j \ge 0$ go to D3.

\item \textbf{D8.} Unnormalize. Now $(q_m\ldots q_1 q_0)$ is the desired quotient. To obtain the
  desired remainder divide $(u_{n-1}\ldots u_1 u_0)$ by $d$ (using a multi-digit right shift
  operation).

\end{enumerate}

\section{Translation Notes}

The ACO library implements extended precision integers using a discriminated type containing an
array to hold the digits. Because of limitations on the way a discriminate can be used, the
array is necessarily indexed starting at one. This conflicts with the description above which
uses a starting index for digits of zero. To interpret the code it is helpful to keep the
following in mind.

First the values of $n$ and $m$ above are the same as used by the code. However, the value of
$j$ that above loops from $m$ down to $0$ is different in the code where it loops from $m + 1$
down to $1$. This means that when arrays are indexed by an expression involving $j$ they already
have an extra $+1$ offset correcting for the different starting index used by the code. However,
arrays that do not use $j$ in their index expressions \emph{may} need an addition $+1$ to offset
the index value from that shown above to that required by the code.

The important exception is for arrays indexed by a loop counter that ranges over
\texttt{Digit\_Index\_Type}. That type also starts at one and such loop counters will thus
contain an ``extra'' $+1$ as well. In the case where both $j$ and a loop counter appear in an
index expression there are two extra $+1$ terms involved and an explicit $-1$ needs to be added
to correct for one of them.

For example if the description of the algorithm above uses an index expression of $j+n$ the code
also uses $j+n$ because the $j$ used by the code is already one more than the $j$ defined in
this document. However if the algorithm uses an index expression of $n-1$ the code uses $n$
since the code needs an extra $+1$ offset for indicies. Similarly an index expression of $n-2$
above is translated to $n-1$ in the code whereas an index expression of $j+n-1$ above is
translated to $j+n-1$ because the $j$ in the code already has the necessary additional offset.

We understand this is more confusing than it needs to be. It would be much easier if Ada allowed
discriminated records with array bounds using expressions. The natural translation of the
algorithm above would be to declare \texttt{Long\_Digits} with bounds of \texttt{0 ..
  Digit\_Length - 1}. Unfortunately the use of \texttt{Digit\_Length - 1} is illegal in that
context.

\bibliographystyle{plain}
\bibliography{references}

\end{document}
